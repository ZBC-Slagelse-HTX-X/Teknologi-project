\section{Produktionsforberedelse}
    \subsection{Website-hosting}
        Der er to muligheder for hosting af websittet.
        Der er fjernhosting, hvor man lejer sig ind hos en central webserver, fx Vercel.
        Der er den anden mulighed, local hosting, hvor vi opstiller en decentral server på fx en skoles grund, hvor man så selv indkøber serverudstyr, veligeholder dette in-house eller igennem indkaldte teknikere\dots

        Det er gratis at anvende services som Vercel indtil et vist serverload opnås, hvorefter det koster penge. Jf. deres hjemmeside\cite{vercelprincing} kan hjemmesiden kun tigås en million gange per måned ved gratis abonnoment, ti millioner ved 20\$ per måned, men da vi forventer, at servicere et konservativt estimat på 50.000 gymnasieelever (baseret på Danmarks statistik \cite{ungudd}) om dagen, vil dette hurtigt blive overskredet, ergo er der tale om en unik dataaftale, hvorfor det nok bliver nok bliver markant dyre, men vil garanteret være billigere i sidste ende.

    \subsection{RFID-løsning /  låsemekanismesystem}
        RFID-låseløsningsmekanismesystemet skal først videreudvikles i Danmark, indtil den er færdigklar, hvorefter produktionen skal udliciteres til en fabrik i Kina.
    \subsection{Teknologianalyse}
        Produktionen og veligholdesen af Lectio-rework applikation skal udliciteres til softwareudviklere fra Indien.
        Som udgangspunk skal hovedudviklingen af Lectio-reworket forgå i Danmark. I Danmark er infrastrukturen helt fin, og den er god nok i Indien til vores formål. Kulturen i Indien
        er meget inbydende til softwarevirksomheder som denne og arbejder gerne for et sådan projekt for den rette løn. Import / eksport til Kina ift. RFID-systemets 
        produktion er særligt nemt, da logistikinfrastrukturen er veletableret og -udbygget. Software kan let eksporteres fra Indien.
