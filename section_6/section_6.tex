\section{Produktionsforberedelse}
    \subsection{Website-hosting}
        Der er to muligheder for hosting af websittet.
        Der er fjernhosting, hvor man lejer sig ind hos en central webserver, fx Vercel.
        Der er den anden mulighed, local hosting, hvor vi opstiller en decentral server på fx en skoles grund, hvor man så selv indkøber serverudstyr, veligeholder dette in-house eller igennem indkaldte teknikere\dots

        Det er gratis at anvende services som Vercel indtil et vist serverload opnås, hvorefter det koster penge. Jf. deres hjemmeside\cite{vercelprincing} kan hjemmesiden kun tigås en million gange per måned ved gratis abonnoment, ti millioner ved 20\$ per måned, men da vi forventer, at servicere et konservativt estimat på 50.000 gymnasieelever (baseret på Danmarks statistik \cite{ungudd}) om dagen, vil dette hurtigt blive overskredet, ergo er der tale om en unik dataaftale, hvorfor det nok bliver nok bliver markant dyre, men vil garanteret være billigere i sidste ende.

    \subsection{RFID-løsning /  låsemekanismesystem}
    \subsubsection{Fremtidig Plan}
    Da vi startede med projektet havde vi meget høje forventninger til hvor meget vi kunne nå at lave, selvom de ikke blev en realitet,
    har vi stadigvæk tænkt mos at dele vores ideer. 
    \newline
    \newline
    Vores oprindelige ide var at bygge hele systemet, både med dørlukker, RFID scanner, og et komplet software system med database der ville havde givet elever og lærere
    mulighed for at booke lokaler og se hvem der var tilstede. Vores plan for at bygge døren var at købe en billig branddør,
    vi ville derefter havde opsæt en "skelet" dør (en dør som ikke sider fast i en væg), så vi ville have haft mulighed for at fremvise, hvordan produktet ville virke i praksis.
    vi havde tænkt os at bestille en elektronisk dørlukker, som vi ville havde koblet op til vores ESP32 i vores RFID kasse,
    vores plan var at trække 12v (Ledninger med 12 volts spænding) for at undgå højstrøms lovgivningen da 12v ledninger ikke indgår under denne lovgivning,
    dette ville havde signifikant sænket omkostninger for opsætning af systemet på skolen. Vi havde tænkt os at benytte et relæ (er et mekanisk kontakt),
    til at styre både dørlukkeren og den lås som vi ville havde sat i dørkarmen for at undgå kompliceret installation i dørene,
    også i forhold til tilfælde, hvor det kunne blive nødvendigt at udskifte døre. Dette ville også gøre designet mere simpelt og nemmere
    for os at konstruere. Desværre nåede vi aldrig så langt med vores planer, dette skyldes en del grunde men vi har tænkt os at nævne
    et par af de lidt større fejl og udfordringer der var skyld i at det ikke blev en realitet.
    \begin{enumerate}
        \item \textbf{For store ambitioner}\
            I starten af projektet havde vi en masse ideer som vi gerne ville føre til livs, dette resulterede i at størstedelen af vores tid blev
            brugt på at finde produkter og undersøge installations metoder ud fra vores meget store ambitioner, dette er jo nødvendigvis ikke en
            dårlig ting, men det var skyld i at vi i det første stykke tid ikke kom i gang med den den indledende konstruktion
            da vi var fokuseret på alle de mange ting vi ville. Da det så endeligt gik op for os at vi var ved at løbe tør for
            tid, havde vi allerede brugt en signifikant mængde af hele projekttiden.
        \item \textbf{For lidt prototyping}\
            Vi brugte rigtigt meget tid på at undersøge og prøve at udvikle et perfekt produkt før vi vidste om det overhovedet ville virke i
            praksis, hvis vi havde fokuseret mere på at lave nogen prototyper og eksperimentere i starten havde vi realiseret nogen af de
            design fejl der senere blev åbenlyse da vi lavede vores første prototype. Dette resulterede i at da vi endelig gik i gang med at
            konstruere og teste vores første prototype var vi nød til at gå tilbage til tegnebrættet og retænke vores design, hvilket
            resulterede i at der var mange ting vi ikke nåede i mål med da vi endeligt havde et fungerende design.
    \end{enumerate}
    
    \subsubsection{Produktion}
    Da vi startede med at konstruere vores prototype, kom vi frem til at det ville være nemmest for os at designe vores initiale prototype ud fra en allerede fungerende guide. Dette gav os mulighed for at lave en virkende
    prototype før vi bevægede os videre med at forbedre prototypen. Som en del af dette lavede vi også en håndfuld skitser som der kan ses nedenfor, de illustrerer vores ide process i perioden hvor vi var i gang med at prototype vores produkt,
    nu kommer der en detajleret gennemgang af hvordan vi konstruerede vores produkt.
    \begin{figure}[H]
        \centering
        \resizebox{!}{13cm}{\includegraphics{assets/design_rfid-system_skitse.jpg}}
        \caption{Endelig Skitse RFID-System}
    \end{figure}
    \newpage
    \begin{enumerate}
        \item \textbf{Sammenkobling af ESP32 og RFID-RC522}
            Det første skridt i vores produktion af vores produkt var at vi skulle have sammenkoblet vores indkøbte ESP32 og RFID-RC522, dette gjorde vi ved at benytte os af et såkaldt 
            breadboard, det er et værktøj der er ekstrem nyttig når man er i gang med en udviklingsprocess da det giver dig mulighed for at sammenkoble dine komponeneter uden at du skal til 
            at lode det sammen, dette er nemlig ekstremt nyttigt fordi det giver dig mulighed for at prøve forskellige sammensætninger af dine komponenter uden af du skal sidde og lode og 
            aflode de forskellige komponenter du bruger. ved hjælp af det her breadboard eksperimenterede vi så med forskellige ideer som vi havde haft, det vi kom til sidst frem til at det 
            ville virke bedst, hvis vi monterede alle vores komponenter på breadboarded med nogle stive ledninger så at vi havde mulighed for at få præcise længder på vores ledninger hvilket 
            gjorde det muligt for os at gøre den interne del af vores produkt mere overskueligt.

        \item \textbf{Strømforsyning} 
            Vi besluttede os for at benytte en strømforsyning(et 3,7v 2600mAh lithium-ion batteri) dette gav os muligheden for at komme vores system ind i en kasse efter at vi havde programmeret
            den med det kode der er nødvendigt for at vise ideen med produktet.


        \item \textbf{Output Wiring}
            I vores endelige protoype benytter vi os af to led dioder til at vise at vores system kan registere RFID-tags og også kende forskel på dem, dette er det der kommer til at give os mulighed 
            for at lave et fraværsystem og lokalebooking, for hvis vi udsteder et elevkort til eleverne på skolen vil vi have mulighed for at kende forskel på hvert enkle elevs unikke kort-id og derved 
            give dem adgang til forskellige lokaler(dem de har booket) og registere dem i fraværssystemet om morgnen nå de tjekker ind med deres kort(eller telefon).

    \end{enumerate}

                                                                         

    \subsection{Teknologianalyse}
        Produktionen og veligholdesen af Lectio-rework applikation skal udliciteres til softwareudviklere fra Indien.
        Som udgangspunk skal hovedudviklingen af Lectio-reworket forgå i Danmark. I Danmark er infrastrukturen helt fin, og den er god nok i Indien til vores formål. Kulturen i Indien
        er meget inbydende til softwarevirksomheder som denne og arbejder gerne for et sådan projekt for den rette løn. Import / eksport til Kina ift. RFID-systemets 
        produktion er særligt nemt, da logistikinfrastrukturen er veletableret og -udbygget. Software kan let eksporteres fra Indien.
