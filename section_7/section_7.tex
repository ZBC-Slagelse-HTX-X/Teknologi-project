\section{Evaluering, vurdering og konklusion \label{sec:konklusion}}
\subsection{Test af produkt}
Vi har testet produktet undervejs ved at udsættet testpersoner for produktet, hvorefter de har vurderet produktets æstetik og brugervenlighed rent kvalitativt, 
hvilket vi har taget til og ageret på.
\subsection{Teknologivurdering}
\subsubsection{Konsekvensvurdering}
Vi har lavet en konsekvensvurdering af Lectio, hvor vi er kommet frem til, at det er en ineffektiv IT-løsning, hvorefter vi har lavet et rework af dette med disse fejl og mangler rettet. 
Om hvorvidt vores applikationspakke bedre respekterer sine brugers tid, dvs. sparer den, vil kræve en kvantiativ undersøgelse, der først kan igangsættes, når produktet er blevet frigivet til flere mennesker. 
Envidere ville en kvalitativ undersøgelse skulle igangsættes for at vurdere om brugervenligheden er forbedret.
\subsection{Procesevaluering}
Vi har anvendt Gantt-diagram, skitser og systematiske metoder til at finde et fyldesgørende produkt til et relevant samfundsmæssigt problem, der vedrører vores studiemiljø. 
Det er gået udemærket, dog har vi skulle hugge en hæl og klippe en tå for at nå i mål--det er overordnet gået meget godt med den tid og de ressourcer, vi har haft til rådighed, dog var projektet en smule for ambitiøst. 
En anden gang vil vi anvende et andet markup-sprog til at holde styr og dokumentere vores projekt i, nemlig Org-Mode fremfor \LaTeX, da dette vil gøre, at vi bedre kan integrere kode i projektet samt fokusere mere på 
andre aspekter af projekter samt tillader os at anvende den overlegne Emacs til sit fulde potentiale.

\subsection{Konklusion}
All-in-all et vellykket projekt.