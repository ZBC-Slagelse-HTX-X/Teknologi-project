\documentclass[12pt, a4paper]{article}
\usepackage{amssymb, amsmath, amsthm} % matematikfunktionalitet
\usepackage[danish]{babel}
\usepackage{lipsum}
\usepackage[margin=2.5cm]{geometry}
\usepackage{parskip}
\usepackage{graphicx} %Titlepage
\usepackage[colorlinks = true,linkcolor = blue,urlcolor  = blue,citecolor = blue,anchorcolor = blue]{hyperref} %Titlepage - ændrer formateringen af links
\usepackage{tikz}
\usepackage{pgfgantt}
\usetikzlibrary{shapes, arrows}
\usepackage{pdflscape}
\usepackage{float}


\title{Projektbeskrivelse}
\author{Jeppe, Alexander, Andreas, David}
\date{}

\begin{document}
\maketitle
\section{Problemanalyse - intro}
    Samfundsmæssigt
    Brandsikkerhed
    Fraværsregistrering 
    modernisering af IT Systemer og forbedret infrastruktur.

\section{Produktudkast}
    Vores Lectio Renovering modernisser HTX' IT-systemer. Det vil vi blandt andet gøre ved at lancere et identitesbaseret system på RFID, hvilket gør en ende på spildt tid med fraværstagning og eventuelt falsk fravær, da dette automatisk registreres, når du scanner dit ID-kort. 
\subsection{Lokale-Booking-System}
    Vores Lokale-Booking-System vil revolutionere måden, hvorpå vi booker lokaler. 
    Det vil gøre en ende på unødigt tidsspild, altså den tid, det tager at gå op på kontoret, frem og tilbage med nøgler.
\subsection{Smartdøre}
    Vores Smartdøre vil forbedre brandsikkerheden og fungerer sammen med booking-systemet og elevernes identifikationskort.
    I tilfælde af brand vil alle skolens døre låse op og lukke automatisk så alle rum bliver isoleret, og brænden bliver forsøgt kvalt. 
\subsection{Lectio Renovering}
    Det vil også sige, at hvis et lokale nu er låst, og uheldet er ude, hvor man skal flygte ud af et vindue, der er bag et aflåst lokale, vil man nu kunne spare potientelt tabte liv.

    \begin{landscape}
\section{Tidsplan}
    \begin{figure}[H]
        \resizebox{\columnwidth}{!}{%
        \begin{ganttchart}{1}{60}
            \gantttitle{2024}{60} \\
            \gantttitle{Marts}{30} \gantttitle{April}{30} \\
            \gantttitlelist{1,...,30}{1} \gantttitlelist{1,...,30}{1}\\
            \ganttgroup{Forberedelse}{1}{7} \\
            \ganttbar{Projektbeskrivelse}{1}{6} \\
            % \ganttlinkedbar{Task 2}{3}{7} \ganttnewline
            % \ganttmilestone{Milestone}{7} \ganttnewline
            % \ganttbar{Final Task}{8}{12}
            % \ganttlink{elem2}{elem3}
            % \ganttlink{elem3}{elem4}
        \end{ganttchart}}
        \caption{Viser Gantt-Diagram over vores foreløbige tidsplan}
    \end{figure}
\end{landscape}
\end{document}

